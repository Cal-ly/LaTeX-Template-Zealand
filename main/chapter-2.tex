\chapter{Udgangspunkt for projektet}
\label{chapter:udgangspunkt}

Den 11. april blev det originale team splittet op. Dette kapitel anvendes til at reflektere over forløbet op til opsplitningen, hvad der skete undervejs og efterfølgende, samt de tilknyttede overvejelser.
Dette kapitel er altså ikke en beskrivelse af selve \emph{SCRUM}-processen, men et udgangspunkt for en ny opstart. Kapitlet danner derfor grundlag for en ny projektplan, der vil blive udarbejdet i de næste kapitler.

\section{Sprint Retrospektive}
For at holde \emph{SCRUM}-rammen, vil jeg bruge et \emph{Sprint retrospektive} til at reflektere over opsplitningen.
Jeg tillader mig dog at tilføje punktet "hvad var planen" til "Hvad gik godt, hvad gik skidt og hvad gør vi til næste gang". Grundet det reflekterende perspektiv, skrives afsnittet i datid.

\subsubsection{Hvad var planen}
Der blev med projektkontrakten sat en ramme for samarbejdet, og teamet har tidligere arbejdet sammen om semesterprojektet i første semester.
Der var en fælles, udtalt forståelse om et højt ambitionsniveau for projektet, og \emph{SCRUM} skulle være grundlaget for selve projektstyringen.
Arbejdsbyrden blev sat til at være indenfor normal undervisningstid på de gængse undervisningsdage (mandag-tirsdag og torsdag-fredag).

\subsubsection{Hvad gik godt}
Businessanalysen med bl.a. \emph{BMC} og \emph{SWOT} blev relativt hurtigt eksekveret.

\subsubsection{Hvad gik skidt}
Der var blandt alle medlemmer, en eller anden form for utilfredshed med de andre i teamet, dog ikke noget der blev luftet åbent. 
Der var tillige friktioner mellem visse teammedlemmer fra sidste projekt, som måske var begyndt i starten af studiet. Utilfredsheden samt friktionerne blev mere og mere tydelige og begyndte at påvirke samarbejdet. 
Eksempelvis blev det nævnt i en sidebemærkning, at det var frustrerende for et teammedlem, at den officielle frokostpause blev sprunget over, fordi et andet teammedlem ikke medbragte eller spiste frokost og derfor ønskede at gå tidligere hjem. 
Efter gruppedannelsen begyndte dele af teamet at forlange, at der kun måtte anvendes koncepter, syntax m.m. som var blevet undervist i eller havde været anvendt i opgaverne i semesteret. 
Der var bl.a. en situation, hvor et forslag om at anvende denormalisering af databasen, ikke engang blev taget op i plenum, før en fra teamet kom med en person-orienteret kommentar.
Friktionerne, den passive-aggressive tone m.m. resulterede i en form for mistillid, der skabte en dårlig stemning for alle teammedlemmer. 
Dette miljø nedbrød enhver konstruktiv dialog, og det førte til konklusionen, at et professionelt samarbejde ikke længere var en mulighed.

Derudover var der en meget forskellig opfattelse af, hvordan \emph{SCRUM} skulle anvendes. Der var teamets ønske, fra sidste projekt, om mere overordnet styring. Dette forsøgtes implementeret i dette projekt, da en \emph{SCRUM master} fremlagde en overordnet tidsplan for en dag, hvor det var nødvendigt, at hele teamet arbejdede sammen. 
Situationen udviklede sig desværre til en person-orienteret kommentar, hvor et teammedlem fortalte \emph{SCRUM master}, at denne skulle stoppe med at prøve på at være projektleder og bestemme over de andre teammedlemmer.

\subsubsection{Hvad gøres anderledes til næste gang}
Der kan spekuleres i, hvor mange af disse udfordringer, der kunne være blevet udglattet eller måske endda undgået, hvis den interpersonelle kemi i teamet havde været bedre. Jeg vil dog stadig nedfælde nogle punkter, der kan anvendes til fremtiden.

Et højt ambitionsniveau kan stå i kontrast til den begrænsede arbejdstid, som teamet fastsatte. Fremover bør der estimeres et mere konkret mål for et projekt og derefter afsættes tid eller vice versa, hvis ønsket om at holde studiet indenfor undervisningstiden vægtes højest.

Den overordnede ramme, i dette tilfælde \emph{SCRUM}, bør også defineres mere klart. 
Selvom \emph{SCRUM} var forsøgt defineret i dette projekt, bl.a. var det beskrevet, at \emph{SCRUM master} skulle facilitere møderne, var der uenighed om, hvad det at \emph{facilitere} indebar. 
Hvis man vælger at have en \emph{SCRUM master}, så bør der være en klar relation mellem ansvar, pligter og beføjelser. 
Hvis \emph{SCRUM master} skal være \emph{process owner}, bør der være klarhed om, at \emph{SCRUM master} bl.a. kan, bør og skal indkalde til møder og sætte dagsorden samt tidsplan for disse. Selv i velfungerende og sammenrystede teams, sammensat af højt motiverede og kompetente medlemmer, er der behov for en med det overordnede ansvar for processen. 
Både teori og erfaring peger mod, at nye, uprøvede teams \cite{mit-newteams} kræver bl.a. klare mål og mere struktur for at lykkes og trives. Dette er yderligere understreget, når de individuelle medlemmer har begrænset erfaring indenfor feltet, de skal arbejde i.

\section{Videreførsel}
Dette afsnit er skrevet kort efter opsplitningen med henblik på at fremstille tanker og planlægning, der er foregået efterfølgende.

\subsection{Opdeling af arv}
Det materiale, der var udarbejdet i fællesskab, er fælles eje, og store dele af det vil gå igen uden citat, bl.a. forretningsanalyse. Andre dele vil dog være markant omskrevet, således at de følger tråden i dette projekt. Rapporten, som den så ud 11. april, er vedlagt som bilag og kan blive refereret ved behov i denne rapport.

\subsection{SCRUM som enkeltmand}
Det kan anskues som kunstigt at skulle videreføre \emph{SCRUM} med alt, hvad det indeholder, når man ikke har en \emph{Product Owner} og er et team bestående af én. Visse \emph{SCRUM}-artefakter kan dog anvendes til projektstyring, bl.a. \emph{User Stories} og \emph{Burndown Charts} bevare sit oprindelige formål. Det vil undersøges, hvorvidt en lektor kan indgå som \emph{Product Owner} til nogle sessioner.

\subsection{Sprint planning}
Tidsrammen fra opsplitningen til deadline giver knap syv uger til at komme i mål. 
Det forventes, at der afsættes en uge til at sammenfatte og redigere eksisterende materiale, bl.a. omskrive og reestimere \emph{User Stories}. Dette vil blive kaldt \emph{Sprint 0}.
Den resterende tid deles op i fem \emph{Sprints} af én uges varighed og en afsluttende halv uge. Hvert \emph{Sprint} sættes til 37 timer. Den afsluttende uge bruges til bl.a. at sammenfatte rapporten.